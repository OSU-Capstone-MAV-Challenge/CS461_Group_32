\documentclass[onecolumn, draftclsnofoot,10pt, compsoc]{IEEEtran}
\usepackage{url}
\usepackage{setspace}


%\usepackage{graphicx}
%\usepackage{epstopdf}
%\epstopdfsetup{update} % only regenerate pdf files when eps file is newer


\usepackage[utf8]{inputenc}
\usepackage[english]{babel}
\usepackage{indentfirst}


\usepackage{geometry}
\geometry{textheight=9.5in, textwidth=7in}

% 1. Fill in these details
\def \CapstoneTeamName{		MAV Challenge}
\def \CapstoneTeamNumber{		32}
\def \GroupMemberOne{			Justin Sherburne}
\def \GroupMemberTwo{			Kaiyuan Fan}
\def \GroupMemberThree{			Yingshi Huang}
\def \CapstoneProjectName{		AHS Micro-Air Vehicle Challenge}
%\def \CapstoneSponsorCompany{	Columbia Helicopters}
\def \CapstoneSponsorPerson{		Nancy Squires, Ph.D.}

% 2. Uncomment the appropriate line below so that the document type works
\def \DocType{
				Technology Review
				}

\newcommand{\NameSigPair}[1]{\par
\makebox[2.75in][r]{#1} \hfil 	\makebox[3.25in]{\makebox[2.25in]{\hrulefill} \hfill		\makebox[.75in]{\hrulefill}}
\par\vspace{-12pt} \textit{\tiny\noindent
\makebox[2.75in]{} \hfil		\makebox[3.25in]{\makebox[2.25in][r]{Signature} \hfill	\makebox[.75in][r]{Date}}}}
% 3. If the document is not to be signed, uncomment the RENEWcommand below
\renewcommand{\NameSigPair}[1]{#1}

%%%%%%%%%%%%%%%%%%%%%%%%%%%%%%%%%%%%%%%
\begin{document}
\begin{titlepage}
    \pagenumbering{gobble}
    \begin{singlespace}
    	  \hfill
        \par\vspace{.2in}
        \centering
        \scshape{
            \huge CS Capstone \DocType \par
            \textbf{\Huge\CapstoneProjectName}\par
			\vspace{1.5in}
            {\large Prepared for}\par
            {\Large\NameSigPair{\CapstoneSponsorPerson}\par}
			\vspace{3pt}
            {\large Prepared by }\par
            Group\CapstoneTeamNumber\par
            % 5. comment out the line below this one if you do not wish to
						%name your team
            \CapstoneTeamName\par
            \vspace{8pt}
            {\Large
                \NameSigPair{\GroupMemberOne}\par
                \NameSigPair{\GroupMemberTwo}\par
                \NameSigPair{\GroupMemberThree}\par
            }
            \vspace{.5in}
        }
        \begin{abstract}
				Technology Review is to organize detail of criterion,
				determine the existing technologies and select the most
				appropriate method. The problems and features can allocate in four parts,
				wireless communication, communication, controller and
				calculation. At the beginning, design the wireless communication.
				Then, decide the communication between micro-helicopter and the control board;
				on the base of competitions rules, UDP, TCP, and Web-Internet.
				These communications consider as possible solutions for connection.
				After building the communication of data,
				the controller of hardware depends on on-market products
				which can match micro-helicopter computer and
				control board. Last but not lease, calculation is an individual feature.
				The computer will self-calculates for moving.
				Camera, software libraries, distance sensor
				and accelerate positioning will provide information
				for driving the micro-helicopter.
				\end{abstract}
    \end{singlespace}
\end{titlepage}
\newpage
\pagenumbering{arabic}
\tableofcontents
% 7. uncomment this (if applicable). Consider adding a page break.
%\listoffigures
%\listoftables
\clearpage

\section{Project Role}

My role is to learn from requirements and help to design and calculate data in our sub-team.
To build a full-autonomous system micro-helicopter to enter the competition.

\section{Wireless Communication}

\subsection{Overview}

According to the requirements from the competition, either a remotely-controlled or full-autonomous system is needed.
No matter remotely-controlled or full-autonomous system,
wireless communication is a valuable choice for micro-helicopter.
Wireless communication is a type of data transmission, which do not need physical wires.
Wireless communication not only can connect with one devices;
it is capable to connect with multiple devices at the same time.
Wireless communication can use the wireless signal to distinct devices
and transfer data correctly to each different device.
Wireless communication is easy to access and convenience compared with other communications.
In wireless communication, there are various choices like
infrared communication, radio, blue-tooth, Wi-Fi, mobile communication and so on.

\subsection{Criteria}
\begin{enumerate}
\item{Wireless communication will be the bridge of exchanging data between micro-helicopter and computer.}
\item{Micro-helicopter will receive data from the wireless communication}
\item{Wireless communication is able to catch data from Micro-helicopter}
\end{enumerate}

\subsection{Potential Choices}

My current knowledge gives me only two options blue-tooth and Wi-Fi.
However, the choices is not enough, so my research has provided more
like infrared communication, radio, mobile communication.

\subsubsection{Infrared communication}


Infrared communication is a high-speed short range wireless communication.\cite{r1}
It needs to transfer data directly from devices to devices, but no need of network.
It is a less cost of devices. Data will not lose easily.
It is a secure way of communication. The range of distance is about ten to thirty meters.
The speed of transferring data is fast, and it is with a maximum speed of hundred Mbps.
However, the communication can only control one device at a time.
Obstacles like doors, walls, bad weather will affect the infrared communication.
Infrared communication will also provide a strong power which can be harmful to human body.
%http://ieeexplore.ieee.org/document/554222/

\subsubsection{Bluetooth Radio}
Blue-tooth\cite{r2} is another short range communication between devices, like phone.
The range of transmission range is within 10 meter in common devices.
The distance range can manage by the power of the senders and the receivers.
Blue-tooth system operates in unlicensed 2.4-GHz frequency.
Nowadays, blue-tooth is a common tool for transferring data.
The normal rate of transmission is 1 Mbps.
It is a tool which has replaced the use of infrared communication.
%wireless communication

\subsubsection{ZigBee}
Zig-Bee\cite{r2} is a tool similar to blue-tooth. It is designed for products which need lower power consumption.
The Zig-Bee can continued work for months or years. Comparing the transfer rate between blue-tooth and Radio
%wireless communication

\subsubsection{Mobile communication}
Cellular systems is the most common communications used by mobile cell.
The mobile uses the signal from the signal towers.
The area of the signal supply from the tower is like a cell.
And there are many signal tower to combine as the whole system.
The team does not have rights to use the signal tower and
the usage of the phone company is very expensive.

\subsubsection{Wi-Fi communication}
Wi-Fi\cite{r3} is defined as an abbreviation fro wireless fidelity. It is common to use for computer networking.
If person has set up Wi-Fi at a place, this person can access the computer network within a certain range of distance.
The signal from the Wi-Fi can get through the walls or doors, if they are made of woods.
The Wi-Fi communication can even send or receive signal from upstairs or downstairs.
The Wi-Fi communication can transmit data at 2.4 GHz frequency. The frequency of the communication can be change by devices.
The average rate of transmission data is about 50 Mbps to 100 Mbps. It is easy to set up in a new place.
however, the securety of the Wi-Fi communication is not as good as other communications, password can prevent strangers receive informations.

\subsection{Discussion}
2.4-GHz frequency is recommended from the rules of the competition.
So that it becomes one of the reasons that the cellular system has been excluded from the other wireless communications.
The second main reason for excluding the cellular system is the budget of using it.
Image processing and video processing are operations which need high-speed data transmit tunnel.
Zig-Bee and Infrared communication are not as fast as what operations are expected.

\subsection{Conclusion}
According to the previous consideration, Wi-Fi and blue-tooth are better choices than others communications.
Both of the communication is easy to purchase on the market and have similar price.
Compare with the transmission of the data, Wi-Fi has a higher speed. As a result, Wi-Fi becomes better choice of communications.

%%%%%%%%%%%%%%%%%%%%%%%%%%%%%%%%%%%%%%%%%%%%%%%%%%%%%%%%%%%%%%%%%%%%%
\section{Communication System}

\subsection{Overview}
Communications between three hardware: micro-helicopter and control board, control board and remote controller.


\subsection{Criteria}
\begin{enumerate}
\item{Micro-helicopter will receive data from the control board and respond to movement.}
\item{Control board will get information from the remote controller which provides data to micro-helicopter for re-position.}
\end{enumerate}

\subsection{Potential Choices}
Depends on the previous choice of Wi-Fi, UDP, TCP and Web-Internet are considered as methods to connect the hardwares.

\subsubsection{UDP}
UDP\cite{r4} stands for User Datagram Protocol and uses of data transferring.
UDP will not acknowledge the receipt of the sending data.
It means a disadvantage that no resends for lost data.
On the contrary, an advantage is that save time without re-sending missing data.
It is typical for immediate data such as online video, audio transmission, online gaming.
Intermittent video or audio is an appearance of Lost data.

\subsubsection{TCP}
TCP\cite{r5} stands for Transmission Control Protocol which is connection-oriented and re-transmittable.
An advantage is able to resend data and acknowledge lost segments. A disadvantage is extending sending time.
TCP is representative of sending guarantee data like Web-Page, files.

\subsubsection{Web-Internet}
Web-Internet is a tool using internet via the web to communicate.
An advantage is that no need for storage in control board and able to calculate data on other devices.
A disadvantage is that all data depend on transferring which requires unhindered signal transmissions.


\subsection{Discussion}
In theory, the computer will need lots of data to calculate distance and movement.
Ensure data received is not as important as using data to calculate more rough distance,
interrupted data is still value to estimate length and size of target and obstacles.
Although 2.4GHz radio might be impacted, this communication is still able to send and receive data between woods and thin walls.

\subsection{Conclusion}
Because of the situations could be happened, UDP can have better performance than TCP.
Continued and various data sending and receiving for saving time to evaluate distance and length of micro-helicopter to the target or obstacles.
Skip of video will be fixed by the different anger of position. Wi-Fi is one of most commonly used communications and efficient in exchanging information.
UDP and WIFI are preferred solutions.


%%%%%%%%%%%%%%%%%%%%%%%%%%%%%%%%%%%%%%%%%%%%%%%%%%%%%%%%%%%%%%%%%%%%%
\section{calculation System}

\subsection{Overview}
Non-human control of micro-helicopter moving, pre-written code for distance calculations and exact movements depend on calculations.

\subsection{Criteria}
\begin{enumerate}
\item{Correct calculations from information}
\item{Using calculations to make movements}
\end{enumerate}

\subsection{Potential Choices}

\subsubsection{Data from sensor}
Distance sensor is a small tool to get a certain range of distances directly.
The advantage to use distance sensor is receiving data directly and no need to wait calculations.
The disadvantage for distance sensor is vibration of the micro-helicopter will affect the precision ratio of the distance.
Incorrect distance might cause a crash to obstacle or unable to pick up the target which will lead to failure.

\subsubsection{Images from camera}
Cameras can provide more information than distance sensor.
However, more data as input will need more calculation.
More calculation will need correspond libraries. And it will need more time and energy to finish the tasks.

\subsubsection{Buildin libraries}
Software libraries will help the computer calculate data. Most of the formula are provided in libraries which is used for calculate.
However, it require more storage and computer burning. It is a factor to slow down the result of calculations.

\subsubsection{accelerate positioning}
It is a method for calculating from the target or obstacles to micro-helicopter.
It need time for computer to process, but all information can be handle by computer it, which mean fewer energy waste.

\subsection{Discussion}
Distance will be a tool to help estimate the distance calculation but not providing correct information.
Accelerate positioning depends on the images from cameras, but not necessary depend on the software libraries.
However, software libraries can reduce the coding error.

\subsection{Conclusion}
It is better to have correct data than not allowable error data which can lead failure.
So using a camera to receive and calculate data is better than receiving direct data from sensor.

%%%%%%%%%%%%%%%%%%%%%%%%%%%%%%%%%%%%%%%%%%%%%%%%%%%%%%%%%%%%%%%%%%%%%


\vspace{2mm}

%\section{References}
\bibliographystyle{IEEEtran}
\bibliography{references}



\end{document}
