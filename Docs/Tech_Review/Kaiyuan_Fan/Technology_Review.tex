\documentclass[letterpaper, 10, draftclsnofoot, onecolumn,compsoc]{IEEEtran}
\usepackage{listings}
%\usepackage{pgfgantt}
\usepackage{underscore}
\usepackage[bookmarks=true]{hyperref}
\usepackage[utf8]{inputenc}
%\usepackage[english]{babel}
\usepackage{indentfirst}
\usepackage{hyperref}
\hypersetup{
    %bookmarks=false,    % show bookmarks bar?
    pdftitle={MAV Competition Software technology review},    % title
    pdfauthor={Kaiyuan Fan},% author
    pdfkeywords={technology review, Capstone, MAV, AHS}, % list of keywords
    colorlinks=true,       % false: boxed links; true: colored links
    linkcolor=blue,       % color of internal links
    citecolor=black,       % color of links to bibliography
    filecolor=black,        % color of file links
    urlcolor=blue,        % color of external links
    linktoc=page            % only page is linked
}%
\def\class{CS 461 - CS Senior Capstone}
\def\term{Fall 2017}

\usepackage{hyperref}
\title{AHS Micro-Air Vehicle Challenge}
\author{ Instructor: Kevin McGrath, Kirsten Winters \\
    Project Sponsor: Nancy Squires, Ph.D.
}
\begin{document}
\null  
\nointerlineskip  % No skip for prev line
\vfill
\let\snewpage \newpage
\let\newpage \relax
\maketitle
\begin{center}
\class\\
\term\\
\huge{Technology Review}\par
\vspace{2mm}
\large{Written by:}\par
\normalsize{Group 32 - Kaiyuan Fan}\par
\vspace{8mm}
\large{\textbf{Abstract:}}\par 
\end{center}
\vspace{2mm}
\normalsize{ This document will introduce my role in the project and discuss three pieces: data collection equipments, video stream interfaces, and communication ways. For each piece, I will give three potential technologies, analyze trade-offs using criteria and provide the best choice.
}

\let \newpage \snewpage
\vfill 
\break % page break

\tableofcontents

\newpage
\section{Project Role}
My role as a member of computer science team in this project is to actively share information with team leader and co-operate with the mechanical and electrical engineering teams. I will mainly work on implementing the autonomous flight of the vehicle.
\section{Data Collection Equipments}
\subsection{Overview}
We plan to design and manufacture a helicopter to participate in the 2018 micro air vehicle challenge competition. We are going to complete a delivery package mission in the competition. The delivery and pickups location areas are roughly known, the exact location of each area of interest is unknown and there has a rectangular obstacle between the base and the delivery area. Therefore, our flying vehicle needs to equip sensors to be able to find the targets, detect the obstacles, stay flight within the mission boundary and search the landing places.


\subsection{Criteria}
\begin{enumerate}
\item{The sensor should be able to collect data to measure the distance to the ground and the objects, the accuracy, and range of measurement should fit the competition environment.}

\item{The weight of sensor should be light enough to match the competition requirement.}

\item{Sensors should provide enough information for both manual fight and autonomous flight.}

\item{Sensors should be easy to find on market and should not be expensive.}
\end{enumerate}


\subsection{Potential Choices:}
The camera sensor is the most common choice for collecting images. By my research, there are various of sensors available on the market and I find the ultrasonic sensor and infrared sensor could be potential choices.
\subsubsection{Camera Sensor}
Camera Sensors can easily capture the images. For autonomous operation, these “images” can be used by video-processing algorithms for recognizing the target, searching the home-base, and operating the hover-hold. For the manual flight, the camera can collect the most significant real-time stream video and send back to the pilot.

\subsubsection{Ultrasonic sensor}
The ultrasonic sensor can emit a given frequency of sound waves and listens to that sound wave bounce back. The sensor can record the elapsed time between the sound wave being emitted and the sound wave bouncing back. Once we know the speed of the sound wave, it is able to calculate the distance between the ultrasonic sensor and the object. A minimum distance from the sensor is required to provide a time delay so that the “echoes” can be interpreted. \cite{r1} Ultrasonic sensor can measure distance without damage and are easy to use and reliable. 

\subsubsection{Infrared sensor}
Infrared sensors use infrared radiation to detect an object. "The main use of infrared sensors in robotics is for obstacle avoidance".\cite{r2} It also can use for measuring distance without touching a surface. Similar to the ultrasonic sensor, but two close infrared sensors measure distance by shining a beam of infrared light and uses a phototransistor to measure the intensity of the light that bounces back. The response speed of infrared sensor can be fast but the range usually under 20 ft.

\subsection{Discussion}
Ultrasonic sensors can be used to control or indicate the position of targets and obstacles. Some objects might not be detected by ultrasonic sensors. This is because some objects are "shaped or positioned in such a way that the sound wave bounces off the object", but are deflected away from the Ultrasonic sensor. It is also possible for the object to be "too small to reflect enough of the sound wave back to the sensor to be detected". \cite{r5} Ultrasonic sensors and infrared sensors can both be used to detect obstacles, but cannot easily to process data to make it suitable for visualization like the camera sensors. Infrared sensors have a relatively short distance measurement. All three types of sensors are cheap and can be easily found at the market, but the camera is the best technology to use for implementation. With the OpenCV support, we can detect the different targets and landing places with their colors. We can even measure the distance to the object if we know the original size of the object. The camera is the essential technology in our implementation.

\subsection{Conclusion}
We are going to implement two cameras in our design. One in front Camera to process stream video and one at the bottom to find targets and the landing places. We also decide to use an ultrasonic sensor at the bottom to detect the height from the ground and measure the boundary range. The helicopter can be stable hover at an altitude and accomplish various tasks with these two technologies.

\vspace{2mm}
\section{Video Stream Interface}
\subsection{Overview}
In this project, we are choosing Raspberry Pi as our helicopter’s board. Raspberry Pi is a powerful board, we are using Raspberry Pi Camera Module to collect and send stream video (image frames) to our base station. It’s necessary to receive the stable and reliable stream video from the helicopter, so the pilot can manual the flight. We need to use an interface to communicate between the camera module to the base station.

\subsection{Criteria}
\begin{enumerate}
\item{The video stream interface should stream video should under a reasonable latency.}

\item{The video stream interface should support the UDP (User Datagram Protocol) for data transmission.}

\item{The video stream interface should support the Raspberry Pi camera module and convey high-quality stream video.}

\end{enumerate}

\subsection{Potential Choices:}
We have been implemented the MJPG-streamer and it worked well. I found there are plenty of choices we can choose, I will discuss two common technologies: VLC media player and RPi-Cam-Web-Interface.

\subsubsection{VLC media player}
VLC media player (commonly known as VLC)\cite{r4} is a free, open-source, and cross-platform media player. VLC is available for desktop operating systems and mobile platforms, including the Raspberry Pi. It's easy to install the VLC media player on the Rasberry Pi board and use it to stream video to another device.

\subsubsection{ MJPG-streamer }
MJPG-streamer \cite{r6} is also a free and open-source interface. It uses command line application that copies JPEG frames from one or more input plugins to multiple output plugins. It can be used to stream JPEG files over an IP-based network from a webcam to various types of browsers and media players that are capable of receiving MJPG streams. It was originally written for embedded devices with very limited resources in terms of RAM and CPU.

\subsubsection{ RPi-Cam-Web-Interface}
RPi-Cam-Web-Interface \cite{r9} is a web interface especially for the Raspberry Pi camera module. It can be used for a wide variety of applications including surveillance and time-lapse photography. It is a free, open-source and highly configurable interface. It can be opened in any browser including the smartphones. The web interface includes motion detection, time-lapse, and video recording.

\subsection{Discussion}
All three interfaces support the user datagram protocol. In a real-time stream, some data loss may be acceptable with the aim of a timely display. VLC is easy to implement but have a relatively high latency compared to other applications. Both MJPG-streamer and RPi-Cam-Web-Interface can send reliable stream video to the base station. There are many other interfaces like pi-camera fits our video stream criteria. MJPG-streamer support various viewports. RPi-Cam-Web-Interface can extend to use motion detection and time-lapse photography\cite{r9}.

\subsection{Conclusion}
We have various choices of the interface to stream video via the Raspberry Pi camera module. We will keep testing the best suitable one at the future time. Furthermore, we will be programming to implement OpenCV to detect the objects and measure distance in the streaming video.

\vspace{2mm}
\section{Communication Ways}
\subsection{Overview}
Our helicopter need collect data and send to the ground station. The communication between helicopter and ground station should be reliable and stable. Communication requires high speed to process stream video, the communication latency and the connectable distance should be in a reasonable range.

\subsection{Criteria}
\begin{enumerate}
\item{The communication technology should maintain a transmit speed capable of two videos streaming at 720p 30fps. This is equivalent to roughly 15Mb/s.}
\item {The communication should be stable under the range of 40 ft.}
\item {The communication technology should follow the competition rule with the standard communication frequency 2.4 GHz.}
\end{enumerate}

\subsection{Potential Choices:}
Our flying vehicle needs be able to remotely communicate with the base station. Wireless communication technology is the prerequisite. The Wi-Fi and Bluetooth are two effective wireless communication technology. While I only find these two technologies, I opened the scope of research and include wired ethernet as the potential choice.
\subsubsection{Wi-Fi}
Wi-Fi, or Wireless Fidelity\cite{r3} is the name of a popular wireless networking technology that uses radio waves to provide wireless high-speed Internet and network connections. Wi-Fi technology are commonly using in our daily life. Wi-Fi allows networking of computers and our flying vehicle without the need for wires. Data is transferred over radio waves, allowing Wi-Fi capable devices to receive and transmit data when they are in the range of a Wi-Fi network. 

\subsubsection{Bluetooth}
Bluetooth is another wireless communication technology. It is can exchange data over short distances by using radio waves at the 2.4GHz band. "By using the 2.4 GHz band, two Bluetooth devices each other can share up to 720 Kbps of capacity".\cite{r8} The connectable range of Bluetooth is relatively low as 10 meters, about 30ft.

\subsubsection{Wired Ethernet}
Ethernet is a network technology commonly used in local area networks. "A wired Ethernet connection can theoretically offer up to 10 Gb/s",\cite{r7}and the transmit speed is consistent. Wired networks are less expensive, faster, and more secure than wireless networks. However, the wired ethernet provides these advantages, it also has its biggest disadvantage, the immobility.

\subsection{Discussion}
Wired communication is incapable for our project while it provides the best transmit speed and stability. Our helicopter must use the wireless communication technology. The Bluetooth cannot give us enough transmit rate compare with the Wi-Fi, the transmission rate can not able to support two videos streaming. The connectable distance of Bluetooth usually under 30ft. Our helicopter will travel and send data as far as 40ft, the connection will be unstable when the distance increases. Wi-Fi can provide much more reliable and stable communication between the helicopter and the ground station. 

\subsection{Conclusion}
We are using Raspberry Pi Zero W which including the wireless network module and Bluetooth. Wi-Fi is the best choice as our communication technology to send and receive data. The network will be provided by the competition area or by our own network router.

\vspace{2mm}
\raggedright
\bibliographystyle{IEEEtran}
\bibliography{references}

\end{document}