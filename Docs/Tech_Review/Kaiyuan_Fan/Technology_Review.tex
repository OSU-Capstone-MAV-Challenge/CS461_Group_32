\documentclass[letterpaper, 10, draftclsnofoot, onecolumn,compsoc]{IEEEtran}
\usepackage{listings}
%\usepackage{pgfgantt}
\usepackage{underscore}
\usepackage[bookmarks=true]{hyperref}
\usepackage[utf8]{inputenc}
%\usepackage[english]{babel}
\usepackage{indentfirst}
\usepackage{hyperref}
\usepackage{url}
\hypersetup{
    %bookmarks=false,    % show bookmarks bar?
    pdftitle={MAV Competition Software technology review},    % title
    pdfauthor={Kaiyuan Fan},% author
    pdfkeywords={technology review, Capstone, MAV, AHS}, % list of keywords
    colorlinks=true,       % false: boxed links; true: colored links
    linkcolor=blue,       % color of internal links
    citecolor=black,       % color of links to bibliography
    filecolor=black,        % color of file links
    urlcolor=blue,        % color of external links
    linktoc=page            % only page is linked
}%
\def\class{CS 461 - CS Senior Capstone}
\def\term{Fall 2017}

\usepackage{hyperref}
\title{AHS Micro-Air Vehicle Challenge}
\author{ Instructor: Kevin McGrath, Kirsten Winters \\
    Project Sponsor: Nancy Squires, Ph.D.
}
\begin{document}
\null  
\nointerlineskip  % No skip for prev line
\vfill
\let\snewpage \newpage
\let\newpage \relax
\maketitle
\begin{center}
\class\\
\term\\
\huge{Technology Review}\par
\vspace{2mm}
\large{Written by:}\par
\normalsize{Group 32 - Kaiyuan Fan}\par
\vspace{8mm}
%\large{\textbf{Abstract:}}\par 
\end{center}
%\vspace{2mm}
%\normalsize{
%}

\let \newpage \snewpage
\vfill 
\break % page break

\tableofcontents

\newpage
\section{Data Collection}
\subsection{Overview}
We are going to design and manufacture a helicopter to compete in the micro air vehicle challenge. We are going to complete a delivery package mission. the delivery/pickups location area is roughly known, the exact location of each area of interest is unknown and may change from team to team. Our helicopter needs able to find the targets, detect the obstacles, stay flight within the boundary and search the landing places.


\subsection{Criteria}
\begin{itemize} 
\item Able to measure the helicopter flight altitude.
\item Able to measure the distance to the obstacles.
\item Able to measure the range of the boundary.
\item Able to find the targets.
\item Able to find the landing places.
\item Able to provide enough information for both manual fight and autonomous flight.
\end{itemize}

\subsection{Potential Choices:}
\subsubsection{Camera Sensor}
Camera Sensors can easily capture the images. For autonomous operation, these “images” can be used by video-processing algorithms for recognizing the target, searching the home-base, and operating the hover-hold. For the manual flight, the camera can send back the most significant real-time stream video to the pilot.

\subsubsection{Ultrasonic sensor}
Ultrasonic sensors can send out a sound wave at a specific frequency and listen for that sound wave to bounce back. By recording the elapsed time between the sound wave being generated and the sound wave bouncing back, it is possible to calculate the distance between the sonar sensor and the object. A minimum distance from the sensor is required to provide a time delay so that the “echoes” can be interpreted. Ultrasonic sensor can measure distance without damage and are easy to use and reliable. \cite{r1}

\subsubsection{Infrared sensor}
Infrared sensors use infrared radiation to detect an object. Infrared radiation is used in industrial, scientific, and medical applications. Night-vision devices using active near-infrared illumination allow people or animals to be observed without the observer being detected. \cite{r2} A thermal IR sensor can measure the heat of an object as well as detects the motion.

\subsection{Discussion}
Ultrasonic sensors can be used to control or indicate the position of objects and materials. Some objects might not be detected by ultrasonic sensors. This is because some objects are shaped or positioned in such a way that the sound wave bounces off the object, but are deflected away from the Ultrasonic sensor. It is also possible for the object to be too small to reflect enough of the sound wave back to the sensor to be detected. \cite{r5} Ultrasonic sensors and infrared sensors can be used to detect obstacles, but cannot easily to process data to make it suitable for visualization. Infrared sensors have a relatively short distance measurement. All three types of sensors can be easily found at the market, but the camera is the best technology to use for implementation. With the OpenCV support, we can detect the different targets and landing places with their colors. We can even measure the distance to the object if we know the original size of the object. The camera is the essential technology in our implementation.

\subsection{Conclusion}
We are going to implement two cameras in our design. One in front Camera to process stream video and one at the bottom to find targets and the landing places. We also decide to use an ultrasonic sensor at the bottom to detect the height from the ground and measure the boundary range. The helicopter will be stable hover at an altitude and accomplish various tasks with these two technologies.

\vspace{2mm}
\section{Video Stream Interface}
\subsection{Overview}
In this project, we are choosing Raspberry pi as our helicopter’s board. Raspberry Pi is a powerful board, we are using Raspberry Pi Camera Module to collect and send stream video (image frames) to our base station. It’s necessary to receive the stable and reliable stream video from the helicopter, so the pilot can manual the flight. We need to use an interface to communicate between the camera module with the base station.

\subsection{Criteria}
\begin{itemize} 
\item  Stream video should under a reasonable latency.
\item  Support the User Datagram Protocol for data transmission.
\item  Support the Raspberry Pi camera module.
\item  Able to provide a high quality stream video.
\end{itemize}
\subsection{Potential Choices:}
\subsubsection{VLC}
VLC media player (commonly known as VLC) is a free and open-source, portable and cross-platform media player and streaming media server developed by the VideoLAN project. VLC is available for desktop operating systems and mobile platforms, including Raspberry Pi. \cite{r4} 

\subsubsection{ mjpg-streamer }
mjpg-streamer is a command line application that copies JPEG frames from one or more input plugins to multiple output plugins. It can be used to stream JPEG files over an IP-based network from a webcam to various types of viewers such as Chrome, Firefox, Cambozola, VLC, mplayer, and other software capable of receiving MJPG streams.It was originally written for embedded devices with very limited resources in terms of RAM and CPU. Its predecessor "uvc_streamer" was created because Linux-UVC compatible cameras directly produce JPEG-data, allowing fast and performant M-JPEG streams even from an embedded device running OpenWRT. The input module "input_uvc.so" captures such JPG frames from a connected webcam. mjpg-streamer now supports a variety of different input devices. \cite{r6}

\subsubsection{ RPi-Cam-Web-Interface}
RPi Cam Web Interface is a web interface for the Raspberry Pi Camera module. It can be used for a wide variety of applications including surveillance, dvr recording, and time-lapse photography. It is highly configurable and can be extended with the use of macro scripts. It can be opened in any browser (smartphones included). \cite{r9} The web interface includes motion detection, time-lapse, and image and video recording.

\subsection{Discussion}
The interface should support the UDP. TCP is about data integrity as against real-time. In a real-time stream, some data loss may be acceptable for the sake of timely display. VLC is easy to implement but have a high latency compare to other application. Both mjpg-streamer and RPi-Cam-Web-Interface can send reliable stream video to the base station. There are many other interfaces like picamera fits our video stream requirement. mjpg-streamer support various viewport including VLC. RPi-Cam-Web-Interface can extend to use motion detection and time-lapse photography.

\subsection{Conclusion}
We could use various interface to stream video via the Raspberry Pi camera module. We will keep testing the best suitable one at the future time. Furthermore, we will be programming to implement OpenCV to detect the objects in the streaming video.

\vspace{2mm}
\section{Communication}
\subsection{Overview}
Our helicopter need collect data and send to the ground station. The communication between helicopter and ground station should be reliable and stable. And require high speed to process stream video, the communication latency and the Connectable distance should be in a reasonable range.

\subsection{Criteria}
\begin{itemize} 
\item 	Must be able to maintain a WiFi connection capable of streaming at minimum: 2 videos at 720p 30fps. This is equivalent to roughly 15Mb/s.
\item 	Must be able to maintain communication at a distance of 40 ft.
\item 	Standard communication (preferred 2.4 GHz).
\item  Under a reasonable latency.
\end{itemize}

\subsection{Potential Choices:}
\subsubsection{Wi-Fi}
Wi-Fi is the name of a popular wireless networking technology that uses radio waves to provide wireless high-speed Internet and network connections. Wi-Fi technology powers most home networks, many business local area networks and public hotspot networks. Wi-Fi technology may be used to provide Internet access to devices that are within the range of a wireless network that is connected to the Internet. The coverage of one or more interconnected access points (hotspots) can extend from an area as small as a few rooms to as large as many square kilometers. \cite{r3}

\subsubsection{Bluetooth}
Bluetooth is the foundation for transformative wireless connectivity. It is a wireless technology standard for exchanging data over short distances (using short-wavelength UHF radio waves in the ISM band from 2.4 to 2.485 GHz from fixed and mobile devices, and building personal area networks.\cite{r7}

\subsubsection{Wired Ethernet(LAN)}
Ethernet is a network technology commonly used in local area networks. A wired Ethernet connection can theoretically offer up to 10 Gb/s, and the speed is consistent. Wired networks are less expensive, faster, and more secure than wireless networks. However, the Ethernet cable that provides these advantages, is also its biggest disadvantage, the immobility.

\subsection{Discussion}
Wired communication is incapable for our project while it provides the best transmit speed and stability. Our helicopter must use the wireless communication technology. The Bluetooth cannot give us enough transmit rate compare with the Wi-Fi, and the connectable distance usually under 10 meters. The helicopter will be moving and could be far more than 10 meters from the ground station. While the Bluetooth is a low-power wireless connectivity technology\cite{r8}, the transmit rate may not able to support video streaming and being unstable when the distance increases. Wi-Fi can provide much more reliable and stable communication between the helicopter and the ground station.

\subsection{Conclusion}
We are using Raspberry Pi zero w which including the wireless LAN and Bluetooth. We are going to using Wi-Fi as our communication technology to send and receive data. The network will be provided by the competition area or by our own network router.

\vspace{2mm}
\raggedright
\bibliographystyle{IEEEtran}
\bibliography{references}

\end{document}