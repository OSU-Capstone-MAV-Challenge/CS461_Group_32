\documentclass[letterpaper, 10, draftclsnofoot, onecolumn]{IEEEtran}
\usepackage{listings}
\usepackage{underscore}
\usepackage[bookmarks=true]{hyperref}
\usepackage[utf8]{inputenc}
\usepackage[english]{babel}
\usepackage{indentfirst}
\usepackage{hyperref} 
\hypersetup{
    %bookmarks=false,    % show bookmarks bar?
    pdftitle={MAV Competition Software Requirements},    % title
    pdfauthor={Kaiyuan Fan, Yingshi Huang, Justin Sherburne },% author
    pdfkeywords={requirements documents, Capstone, MAV, AHS}, % list of keywords
    colorlinks=true,       % false: boxed links; true: colored links
    linkcolor=blue,       % color of internal links
    citecolor=black,       % color of links to bibliography
    filecolor=black,        % color of file links
    urlcolor=blue,        % color of external links
    linktoc=page            % only page is linked
}%
\urlstyle{same}
\def\myversion{ 1.0 }
\def\class{CS 461 - CS Senior Capstone}
\def\term{Fall 2017}
\date{11/12/2017}

\usepackage{hyperref}

\title{AHS Micro-Air Vehicle Challenge}
\author{ Instructor: Kevin McGrath, Kirsten Winters \\
    Project Sponsor: Nancy Squires, Ph.D.}
\begin{document}
\null  % Empty line
\nointerlineskip  % No skip for prev line
\vfill
\let\snewpage \newpage
\let\newpage \relax
\maketitle
\begin{center}
\class\\
\term\\
\huge{Technology Review}\par
\vspace{2mm}
\large{Written by:}\par
\normalsize{Justin Sherburne}\par
\normalsize{Group 32: CS Team Lead} \par
\vspace{8mm}
\end{center}

% \large{\textbf{Abstract:}}\par 
% \vspace{2mm}
% \normalsize{ 
% }

\let \newpage \snewpage
\vfill 
\break % page break

\tableofcontents


\section*{Revision History}

\begin{center}
    \begin{tabular}{|c|c|c|c|}
        \hline
	    Name & Date & Reason For Changes & Version\\
        \hline
	    Technology Review & Nov 12 & Initial Creation & 1.0\\
        \hline
    \end{tabular}
\end{center}



\section{Hardware Controllers}

\subsection{Overview}

We were asked by the Electrical and Mechanical teams for this 
project to decide what hardware we would require on-board of our 
Micro-Air Vehicle. This is going to be the brains that directly 
interfaces with the motors. At minimum it should be able to 
translate manual commands into usable flight instructions for the 
vehicle. At most this hardware should be able to handle video and 
data streaming from a base station while maintaining a stable 
flight path. 

\subsection{Criteria}
\begin{enumerate}
\item{The controller should  be smaller than half of our maximum 
allowed dimension. This means it should be shorter than 8.85 inches 
in any dimension. However, smaller boards should be given 
preference in order to reduce our footprint on the overall design.}

\item{The controller should not use an excessive amount of power, 
and should not require an external power adaptor. }

\item{The controller should have a port to attach to a camera. This 
can be USB, PCIe, CSI, I/O pins, ect.}

\item{The controller should have I/O pins capable of both 
transmitting and receiving electrical signals}

\item{The controller should be capable of maintaining a stable 
wireless connection via either built-in hardware or using an 
additional adaptor.}
\end{enumerate}

\subsection{Potential Choices}

My initial research led me to two very similar options: The 
Raspberry Pi, and the OpenRex. For a third option I am including 
the Arduino platform due to its versatility. Another possible third 
option I was considering was the Nvidia Jetson TX1. However, the 
Nvidia was the largest of the three options, and also required the 
most amount of power. 

\subsubsection{Arduino}


The Arduino platform\cite{r1} is unique in the fact that it can be 
customized for most applications you could use a Pi or OpenRex on. 
Adding additional modules called shields to the base Arduino can 
give it wireless capabilities as well as USB and camera input 
options. Additionally, the MKR ZERO is a similar size to the PiZero 
which makes it ideal for our application. Power is supplied via a 
micro-usb input, and runs on less than 2A. Additional boards may 
require separate power, however they can be powered in parallel 
with micro-usb adaptors with the same power supply. I/O pins would 
need to be utilized for additional shields added to the Arduino, 
however a specific shield could also be used for motor control. The 
Arduino has a micro-controller rather than a processor, meaning 
that it cannot support a true OS, and will generally be inferior at 
complex tasks. 


%https://store.arduino.cc/arduino-mkrzero

\subsubsection{Raspberry Pi Model 3}

The Raspberry Pi is one of the most commonly used controller boards 
because it is capable of running a full OS like Raspbian. It's 
power works identically to the Arduino platform, being supplied via 
micro-usb. In addition to being able to support an OS, the Pi also 
comes with additional ports built into the board. The CSI port is 
capable of handling a video feed from a camera, and USB ports can 
also accept camera inputs. There are two versions of the Pi that 
are suitable for our project; the Pi model 3 \cite{r2}, and 
the Pi Zero W \cite{r3}. The Pi Zero is a smaller version of the 3, 
with less processing power, but still maintaining the on-board 
wireless capabilities. 
The processor is reduced on the Pi Zero, which is perhaps the 
largest difference between the two controllers. Both models also 
include 20 I/O pins that can be configured as needed. The Pi 3 
model B has 4 core processor running at 1.2GHz, however it only 
supports 1gb of ram. The Pi Zero has a single core processor 
running at 1GHz, in addition to 512MB of RAM.   


%https://www.raspberrypi.org/products/raspberry-pi-3-model-b/

%https://www.raspberrypi.org/products/raspberry-pi-zero-w/

\subsubsection{OpenRex}


The OpenRex\cite{r4} board is essentially a more-powerful version 
of the Pi 3. It includes a couple more ports like SATA as well as 
an on-board gyroscope. 
While some of these features could prove useful to the 
overall project, others are completely unnecessary for our 
application (like the humidity sensor). Additionally, the OpenRex 
does not have as much community support as the Pi and Arduino 
platforms. I/O, power, and size are nearly identical to the Pi, 
however, the OpenRex does not have built-in wireless communication. 
The OpenRex has a MX6 processor running 4 cores at 1.2GHz. 
Additionally, it supports up to 4gb of ram which means it could be 
capable of handling image processing without a base-station.  


%http://www.imx6rex.com/open-rex/

\subsection{Discussion}

The Pi is the strongest contender here for a number of reasons. The 
first being cost. While the Arduino is very comparable at \$35, the 
OpenRex board costs as much as \$335 for their top model. The Pi 
comes with a similar processor, and still includes a processor in 
its smaller form on the Pi Zero. The Pi Zero costs only \$10, and 
still includes the wireless capabilities that the Arduino and 
OpenRex do not have. The additional boards needed to run the 
Arduino and OpenRex add unnecessary complexity to the project. 

\subsection{Conclusion}

The Pi Zero W is going to be our board of choice for this project. 
While it's processor is smaller than the Pi or OpenRex, our 
solution is to use the processing power of our base-station to our 
advantage. The small form factor reduces our weight and size while 
still keeping the camera, wireless, and I/O connections we need. In 
addition to being small, it is also the cheapest option and has a 
very large support community should we run into problems that are 
hardware specific. 



%%%%%%%%%%%%%%%%%%%%%%%%%%%%%%%%%%%%%%%%%%%%%%%%%%%%%%%%%%%%%%%%%%%%%


\section{Operating Systems}

\subsection{Overview}

For our project we will need 2 operating systems that are capable 
of interacting with each other in one form or another. One OS is for 
our controller on-board of the vehicle, and the second is for our 
base-station. 

\subsection{Criteria}

Operating systems will be evaluated based on the following criteria:
\begin{enumerate}
\item{The OS being loaded onto our controller must be lightweight 
and efficient, capable of running on a Pi Zero W}

\item{The OS running on our base-station must be capable of 
effectively utilizing the processing power of that system, in 
addition to being lightweight and stable.}

\item{Both operating systems should be based on a common platform 
and share similar file structures.}
\end{enumerate}

\subsection{Potential Choices}

There are quite a few Linux distributions floating around that 
could be very applicable to this project. However, Operating 
systems that are widely used will take precedence here because 
generally, they are stable over a variety of platforms. Because of 
this we will be looking into Raspbian for the Pi, as well as Win 10 
iot core. For the base-station we will be comparing Ubuntu to 
Windows architectures. It is important for these to be similar 
because it gives us flexibility during development. If we are 
running Windows we can develop an application on either device 
without needing to translate code for a different system.

\subsubsection{Rasbian}

Rasbian\cite{r5} is a Debian based operating system that is specifically 
designed for the Raspberry Pi boards. Since Pi's have been selected 
as our hardware of choice, it is a logical choice to use Raspbian as 
our default OS. The problem with Raspbian is that it is loaded with 
a lot of unnecessary programs that will not be needed for our 
project. The GUI is beneficial for troubleshooting, however it 
consumes resources and will ultimately be disabled to favor 
performance. 

% https://www.raspbian.org/RaspbianAbout

\subsubsection{Windows 10 / Win 10 iot Core}

Windows 10 and Win 10 iot core\cite{r6} are both based on the Windows 
architecture. The Win 10 core is capable of running on the Pi, and 
interfaces smoothly with other windows devices. Windows is stable 
and can effectively manage hardware resources for bulky programs. 
The downside to this is that there is a small community that is 
using the Win 10 core on the Pi, but support is limited. 
Additionally, if the Pi is not going to be running a Windows OS, 
then the base-station should be similarly converted. 

%https://developer.microsoft.com/en-us/windows/iot

\subsubsection{Ubuntu}

Ubuntu\cite{r7} is also Debian based, like Raspbian. This makes it a 
preferable choice over windows if the Pi is not running a Win 10 
iot core. Ubuntu is the most installed linux distro, and has a 
stable version available for free. There are also lightweight 
versions of Ubuntu, making it a possible candidate for running on 
the Pi as well. Ubuntu would allow us to easily communicate, 
develop, and test using one stable platform.  

%https://www.ubuntu.com/

\subsection{Discussion}

The key in this selection is finding a balance between two 
operating systems that results in simplicity. There are two clear 
options here. The first is a Windows 10 base-station combined with 
a Win 10 iot core running on the Pi. The second option is some sort 
of Debian - Debian configuration. This could mean Ubuntu to Ubuntu 
or Rasbian to Rasbian


\subsection{Conclusion}

It comes as little surprise that Raspbian is going to be the 
preferred OS for the Pi. It is lightweight enough to run 
efficiently on the Pi Zero, even with a GUI. The community support 
for Raspbian also gives it an edge over Win 10 iot core and Ubuntu. 
For our base station we should use Ubuntu, or a similar Debian 
based OS. Ubuntu is stable, widely supported, and can run alongside 
Windows on the same machine. Ubuntu will allow us to develop 
programs for the Pi without needing a special IDE. Additionally, 
using 3rd party utilities like OpenCV are simpler for Ubuntu / 
Raspbian because they are designed for a linux system.  





\section{Image Processing}

\subsection{Overview}

Our Micro-Air Vehicle is required to have an on-board camera system 
to stream video back to our base-station. So we are choosing to use 
this camera with an image-recognition software that is capable to 
finding the objects we need to look for during the competition. 
Image processing software is cumbersome, and with our Pi Zero we 
needed a unique solution to provide adequate processing power. So 
we are choosing to send the video feed back to our base station for 
processing, and sending the corresponding commands back to the 
controller. 

\subsection{Criteria}

\begin{enumerate}
\item{The program or software must be capable of running on a 
laptop that is not connected to the internet.}

\item{It must be able to identify the main obstacle located near 
the center of the competition area.}

\item{It must be able to identify the boundary lines that mark the 
edge of the competition area.}

\item{It must be capable of identifying landing areas, as well as 
the packages we need to pick up and deliver. }
\end{enumerate}

\subsection{Potential Choices}

Writing our own algorithm to locate objects is far beyond the scope 
of this project, so we need a third-party program for the base 
architecture. Matlab can be fairly lightweight if you do not 
include the GUI and additional packages, which makes it a viable 
option for this project. OpenCV and OpenSLAM are open source 
projects that could also meet our criteria. OpenCV is specifically 
designed for image processing, while OpenSLAM is intended for 3-D 
mapping projects. 

\subsubsection{Matlab - Image Processing Toolbox}

Matlab has many different packages that run with it. One of these 
packages is the Image Processing Toolbox\cite{r9}. This toolbox allows you to take a camera or video input and filter out colors, detect 
objects, and digitally stabilize video streams. The downside to 
Matlab is that it is bundled with other packages and isn't 
necessarily the most efficient processing option. For example, the 
image processing toolbox recommends being installed alongside six 
additional toolboxes. It does, however, run on a variety of 
platforms and languages making it simple to migrate from one device 
to another in the future. 

%https://www.mathworks.com/products/image.html

\subsubsection{OpenCV}

OpenCV is an open source library the is designed specifically for 
computer vision and image processing. It runs in C, C++, Python, 
and Java. It also supports Windows, Linux, MacOS, iOS, and Android 
platforms. "OpenCV was designed for computational efficiency and 
with a strong focus on real-time applications. Written in optimized 
C/C++, the library can take advantage of multi-core processing." \cite{r8}.
OpenCV is capable of filtering colors to identify specific objects. 
Additionally, by knowing a size of a specific object, OpenCV can 
then calculate its distance away from that object. 

%https://opencv.org/

\subsubsection{OpenSLAM}

SLAM stands for simultaneous localization and mapping. 
OpenSLAM\cite{r10} is an open-source repository of projects that 
utilize various SLAM algorithms to map environments. Essentially 
SLAM sets out to map a complete environment by tracking an objects 
location in approximation with a moving target. This requires an 
accurate measure of the vehicles speed, as well as a power computer 
to run the algorithm. While SLAM algorithms can be tailored to run 
on the available resources, it is primarily designed for 
applications like autonomous cars which can be fitted with powerful 
processing units. Additionally, it is intended to map 3 dimensional 
environments, not to filter for individual colors.

%http://openslam.org/

\subsection{Discussion}

Matlab provides a robust system that can also be used for other 
sensors and flight information. However, it is not the best option 
for image processing due to its required dependencies. OpenCV and 
OpenSLAM are both open source options that are fully capable of 
identifying the objects we need during the competition. OpenSLAM 
takes this a step further though and actively maps the environment 
we are flying in. 

\subsection{Conclusion}

OpenCV is going to be our preferred option here because it 
accomplished everything that the Matlab software can in a smaller 
package. OpenSLAM could be modified to locate colors during our 
competition, but the 3-D mapping functionality that it is intended 
for is not something we need to have for our project. OpenCV is the 
lightweight and simple option that accomplishes what we need it to 
do and nothing more. 

\vspace{2mm}

%\section{References}
\bibliographystyle{ieeetr}
\bibliography{references}



\end{document}

