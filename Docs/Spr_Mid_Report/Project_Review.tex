\documentclass[onecolumn, draftclsnofoot,10pt, compsoc]{IEEEtran}
\usepackage{url}
\usepackage{setspace}
\usepackage{graphicx}
\usepackage{epstopdf}
\epstopdfsetup{update} % only regenerate pdf files when eps file is newer
\usepackage[utf8]{inputenc}
\usepackage[english]{babel}
\usepackage{indentfirst}
\usepackage{geometry}
\usepackage{color}
\usepackage{tikz}
\usepackage{rotating}
\usepackage{pgfgantt}
\usepackage{xcolor}
\geometry{textheight=9.5in, textwidth=7in}


% 1. Fill in these details
\def \CapstoneTeamName{		MAV Challenge}
\def \CapstoneTeamNumber{		32}
\def \GroupMemberOne{			Justin Sherburne}
\def \GroupMemberTwo{			Kaiyuan Fan}
\def \GroupMemberThree{			Yingshi Huang}
\def \CapstoneProjectName{		AHS Micro-Air Vehicle Challenge}
%\def \CapstoneSponsorCompany{	Columbia Helicopters}
\def \CapstoneSponsorPerson{		Nancy Squires, Ph.D.}

% 2. Uncomment the appropriate line below so that the document type works
\def \DocType{		%Problem Statement
				%Requirements Document
				%Technology Review
				%Design Document
				Midterm Report
				}

\newcommand{\NameSigPair}[1]{\par
\makebox[2.75in][r]{#1} \hfil 	\makebox[3.25in]{\makebox[2.25in]{\hrulefill} \hfill		\makebox[.75in]{\hrulefill}}
\par\vspace{-12pt} \textit{\tiny\noindent
\makebox[2.75in]{} \hfil		\makebox[3.25in]{\makebox[2.25in][r]{Signature} \hfill	\makebox[.75in][r]{Date}}}}
% 3. If the document is not to be signed, uncomment the RENEWcommand below
\renewcommand{\NameSigPair}[1]{#1}

%%%%%%%%%%%%%%%%%%%%%%%%%%%%%%%%%%%%%%%
\begin{document}
\begin{titlepage}
    \pagenumbering{gobble}
    \begin{singlespace}
    	\includegraphics[height=4cm]{coe_v_spot1}
        \hfill
        % 4. If you have a logo, use this includegraphics command to put it on the coversheet.
        %\includegraphics[height=4cm]{CompanyLogo}
        \par\vspace{.2in}
        \centering
        \scshape{
            \huge CS Capstone \DocType \par
            {\large\today}\par
            \vspace{8pt}
            \textbf{\Huge\CapstoneProjectName}\par
			\vspace{1.5in}
            {\large Prepared for}\par
            % \Huge \CapstoneSponsorCompany\par
            % \vspace{5pt}
            {\Large\NameSigPair{\CapstoneSponsorPerson}\par}
			\vspace{3pt}
            {\large Prepared by }\par
            Group\CapstoneTeamNumber\par
            % 5. comment out the line below this one if you do not wish to name your team
            \CapstoneTeamName\par
            \vspace{8pt}
            {\Large
                \NameSigPair{\GroupMemberOne}\par
                \NameSigPair{\GroupMemberTwo}\par
                \NameSigPair{\GroupMemberThree}\par
            }
            \vspace{.5in}
        }
        \begin{abstract}
        The purpose of this document is to provide a reflection on the progress of the Micro Air Vehicle project. We are now twenty six weeks into the project, approximately two weeks from expo. Here we will outline problems and possible solutions we have encountered this far, as well as a breakdown of weekly progress made thus far.
        \end{abstract}
    \end{singlespace}
\end{titlepage}
\newpage
\pagenumbering{arabic}
\tableofcontents
% 7. uncomment this (if applicable). Consider adding a page break.
%\listoffigures
%\listoftables
\clearpage


\section*{Revision History}

\begin{center}
    \begin{tabular}{|c|c|c|c|}
        \hline
		Name & Date & Reason For Changes & Version\\
        \hline
		Midterm Progress Report & May 1, 2018 & Initial Creation & 1.0\\
		\hline
    \end{tabular}
\end{center}




\section{Purpose}

The purpose of this project is to design a vehicle capable of competing in the American Helicopter Society’s Micro-Air Vehicle challenge. We have elected to compete in the autonomous category, meaning our vehicle must be able to fly without any user interaction. We are representing Oregon State University for the first time at this competition, and if our project is successful we could gain additional outside funding similar to the OSU rocketry and solar teams.

\section{Project Goals}
The goal is to create a vehicle capable of navigating the competition environment without human control. Additionally, the vehicle should be able to pick up letters and deliver them to other locations within the competition environment. There are additional constraints on the size, weight, and specific functionalities of the vehicle, but from the computer science standpoint our goals are:

\begin{enumerate}
\item{The vehicle must be able to stream one camera feed to the base station for manual control. }
\item{We must have an emergency cut-off switch in case of a loss of communication or control. }
\item{We must have a manual override that will shut-down autonomous controls.}
\item{Ultrasonic sensors will be used to calculate distance from objects within the competition area in conjunction with the camera.}
\item{Image processing at minimum should be able to identify three objects: The letter, the landing area, and the boundary lines.}
\item{Any flight changes should originate from the base station, and motor controls should be implemented on the vehicle.}
\item{Our vehicle should fully comply with AHS competition rules and guidelines. }
\end{enumerate}



\section{Justin Sherburne's Report}

\subsection{Current Progress}

\subsection{Future Goals}

\subsection{Problems}

\subsection{Solutions}







\section{Yingshi Huang's Report}

\subsection{Current Progress}

\subsection{Future Goals}

\subsection{Problems}

\subsection{Solutions}





\section{Kaiyuan Fan's Report}

\subsection{Current Progress}

\subsection{Future Goals}

\subsection{Problems}

\subsection{Solutions}








\section{Conclusion}




\bibliographystyle{IEEEtran}
\bibliography{references}


\end{document}
