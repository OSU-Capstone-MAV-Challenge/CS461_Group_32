\documentclass[letterpaper,10pt,draftclsnofoot,onecolumn]{IEEEtran}

\usepackage[margin=0.75in]{geometry}

\def\name{Kaiyuan Fan}
\def\class{CS 461 - CS Senior Capstone}
\def\assign{Problem Statement}
\def\term{Fall 2017}
% Title page information
\title{\assign}
\author{\name}
\setlength{\parindent}{4em}
\setlength{\parskip}{1em}
\begin{document}
\maketitle
\centering
\class\\
\term
\vspace*{4em}
\begin{abstract}
American Helicopter Society invites student teams to participate in the 5th Annual Micro Air Vehicle (MAV) Student Challenge. Student teams choose to design either a remotely controlled system or a fully-autonomous system. The aircraft needs are well-designed and able to deliver packages, there will be a competition hold in May 2018 Phoenix. This is the first time Oregon State University going to participate this competition. The project will be mentored and sponsored by Columbia Helicopters. This project will be a collaboration of electrical, mechanical engineering and computer science Capstone students. In this Paper, I will give a description of Micro Air-Vehicle Challenge, address proposed solution, and explain the desired outcomes for this challenge.
\end{abstract}

\clearpage
\begin{flushleft}

\section{Background}
The American Helicopter Society (AHS) International is the world’s largest technical society dedicated to enhancing the understanding of vertical flight technology. The Annual Micro Air Vehicle (MAV) Student Challenge is a competition is sponsored by a number of industry representatives, including Sikorsky Lockheed-Martin, Bell Helicopters, Boeing and Pratt Whitney. AHS invites student teams to participate in. This electric powered vertical take-off and landing (VTOL) MAV competition focuses on the form and function of the MAV. Teams compete with unique, innovative and robust VTOL MAV designs and demonstration of best flight and autonomy capabilities. A key part of the competition is system integration of vehicle flight systems and sensors. The competition takes place in May 2018 in Phoenix.

\section{Description}
The problem we are going to solve is to design and manufacture an electric-powered vertical take-off and landing (VTOL) Micro Air Vehicle. The aircraft is remotely controlled and have the function to deliver packages. There will be an indoor package delivery mission during the competition. The aircraft should be able to steady-state hover, avoid obstacles, and recognize the dropoff point. The vehicle restrictions as below :
\begin{itemize}
\item Vertical takeoff and landing (VTOL) as well as hover capability
\item Any number of rotors/propellers
\item Onboard flight-stabilization 
\item Onboard camera(s) needed for mission – Multiple cameras are allowed.
\item Standard communication (preferred 2.4 GHz)
\item Weight less than 500g (17.6 oz) including batteries (not including the delivery packages)
\item Size less than 45 cm (17.7 inches) in any dimension
\end{itemize}
\par


We are creating an innovative-designed, stable and reliable air vehicle. The aircraft must be fully tested to ensure each function works perfectly. We also need to guarantee the aircraft can adapt with bad weather and unexpected events. A safety pilot is required to override the autonomous systems in case of unacceptable aircraft behavior.\par

Besides the implementation of Micro Air Vehicle, we should design a presentation with a poster about 5 to 10 minutes. The presentation is supposed to illustrate the design process of the air vehicle, explain autonomy elements and package delivery elements. Presentation needs include a short flight demo to present the capabilities of the aircraft for the audience.

\section{Proposed Solution}
We are going to design and manufacture a quadcopter to address the problem. quadcopters are cheaper and more durable than conventional helicopters due to their mechanical simplicity. Their smaller blades are also advantageous because they possess less kinetic energy, reducing their ability to cause damage. For small-scale quadcopters, this makes the vehicles safer for close interaction. It is also possible to fit quadcopters with guards that enclose the rotors, further reducing the potential for damage.\par

Quadcopters differ from conventional helicopters, which use rotors that can vary the pitch of their blades dynamically as they move around the rotor hub. Quadcopters generally use two pairs of identical fixed pitched propellers; two clockwise and two counterclockwise. These use independent variation of the speed of each rotor to achieve control. By changing the speed of each rotor, it is possible to specifically generate a desired total thrust; to locate for the center of thrust both laterally and longitudinally; and to create a desired total torque, or turning force. quadcopters are cheaper and more durable than conventional helicopters due to their mechanical simplicity.\par

As the Computer Science team in this project, we are going to coding the quadcopter system with sufficient functionalities and analysis the data that collecting from sensor and camera embedded in the aircraft. We have responsibility to communicate and collaborate with the electrical engineering and mechanical engineering team.


\section{Desired Outcome}
We will draft a design proposal before the end of the January. The framework of aircraft will be designed and manufactured this fall term. Winter term we will focus on constructing the main function of the aircraft. The testing and debugging will be covered at the same time. Before mid of the March, the test video should be come out. If our team is selected, we will attend the final competition in May at Phoenix. At this moment, the project will be considered as 90 percent completed. After the competition, we have to attend the undergraduate engineer Expo to share the experiences and present the design and manufacture processes.




\section{Reference}
MAV Student Challenge 2017. (n.d.). Retrieved October 10, 2017, from https://vtol.org/education/micro-air-vehicle-student-challenge/micro-air-vehicle-student-challenge-2017\\
Quadcopter. (2017, October 07). Retrieved October 10, 2017, from https://en.wikipedia.org/wiki/Quadcopter

\clearpage
\end{flushleft}
\end{document}

