\documentclass[onecolumn, draftclsnofoot,10pt, compsoc]{IEEEtran}
\usepackage{graphicx}
\usepackage{url}
\usepackage{setspace}

\usepackage{geometry}
\geometry{textheight=9.5in, textwidth=7in}

% 1. Fill in these details
\def \CapstoneTeamName{		}
\def \CapstoneTeamNumber{		32}
\def \GroupMemberOne{			Kaiyuan Fan}
\def \GroupMemberTwo{			Yingshi Huang}
\def \GroupMemberThree{			Justin Sherburne}
\def \CapstoneProjectName{		American Helicopter Society Micro-Air Vehicle Challenge}
\def \CapstoneSponsorCompany{	Columbia Helicopters}
\def \CapstoneSponsorPerson{		}

% 2. Uncomment the appropriate line below so that the document type works
\def \DocType{		Problem Statement
				%Requirements Document
				%Technology Review
				%Design Document
				%Progress Report
				}

\newcommand{\NameSigPair}[1]{\par
\makebox[2.75in][r]{#1} \hfil 	\makebox[3.25in]{\makebox[2.25in]{\hrulefill} \hfill		\makebox[.75in]{\hrulefill}}
\par\vspace{-12pt} \textit{\tiny\noindent
\makebox[2.75in]{} \hfil		\makebox[3.25in]{\makebox[2.25in][r]{Signature} \hfill	\makebox[.75in][r]{Date}}}}
% 3. If the document is not to be signed, uncomment the RENEWcommand below
%\renewcommand{\NameSigPair}[1]{#1}

%%%%%%%%%%%%%%%%%%%%%%%%%%%%%%%%%%%%%%%
\begin{document}
\begin{titlepage}
    \pagenumbering{gobble}
    \begin{singlespace}
    	\includegraphics[height=4cm]{coe_v_spot1}
        \hfill
        % 4. If you have a logo, use this includegraphics command to put it on the coversheet.
        %\includegraphics[height=4cm]{CompanyLogo}
        \par\vspace{.2in}
        \centering
        \scshape{
            \huge CS Capstone \DocType \par
            {\large\today}\par
            \vspace{.5in}
            \textbf{\Huge\CapstoneProjectName}\par
            \vfill
            {\large Prepared for}\par
            \Huge \CapstoneSponsorCompany\par
            \vspace{5pt}
            {\Large\NameSigPair{\CapstoneSponsorPerson}\par}
            {\large Prepared by }\par
            Group\CapstoneTeamNumber\par
            % 5. comment out the line below this one if you do not wish to name your team
            \CapstoneTeamName\par
            \vspace{5pt}
            {\Large
                \NameSigPair{\GroupMemberOne}\par
                \NameSigPair{\GroupMemberTwo}\par
                \NameSigPair{\GroupMemberThree}\par
            }
            \vspace{20pt}
        }
        \begin{abstract}
        % 6. Fill in your abstract
    	Our project is to design a quadcopter which can be light (less than 500g) and fly stably (sensors can send information back fast and continues). A quadcopter can be in a place which provides better vision for remote monitoring. They usually use in some dangerous situations like radiation area, landmines area, crime scene so that people can lower the risks. A quadcopter is usually remotely controlled small electric powered motor. Electrical and mechanical engineering will design what material to use. As computer science students, we will connect quadcopter and controller. Make sure the data which received from the sensors will deliver to the software. Study and compare information from open source lab, review competitions' awards from previous Races. Our goal is to provide a remote control quadcopter on May 2018.
        	\end{abstract}
    \end{singlespace}
\end{titlepage}
\newpage
\pagenumbering{arabic}
\tableofcontents
% 7. uncomment this (if applicable). Consider adding a page break.
%\listoffigures
%\listoftables
\clearpage

% 8. now you write!
\section{The First Section}
A quadcopter is an unmanned helicopter has multiple rotors and usually with video record feature. It is original uses in the military. However, people start to use for many other fields like agriculture. The aerial view of the camera from a quadcopter is designed to have clear footage. A distinct video depends on stable take off, flying period and landing. Vertical vision can provide more options from a start place to a destination.  
\section{The Second Section}
Clearness images can provide detail and immediate feedback of an area or situations. Engineers can analyze feature and input so that people have a better update from real time. Requirements for recording video should not become a burden of a quadcopter. Engineers can catch and optimize relevant information from video easily. Real-time image is giving engineers the exact location and what kinds of conditions the quadcopter is. Apparently, vision support can help the pilots and the flying quadcopter to avoid hurting anything and steer by various any obstructions including living animals so that the mission will not be interrupted. The real-time record will become one of the best assistants of the quadcopter. 
\section{Problem description}
In Micro Air Vehicle (MAV) Student Challenge competition, each entrance quadcopter requires by nine qualifications. First, it can take-off vertically and landing on the ground without crashing or caused by any damage. Second, numbers of tools as rotors or propellers need to become the factors of flying. Third, onboard flight-stabilization is part of the quadcopter. Fourth, vision access via cameras are using for missions. Fifth, communication is transfer from the radio transmitter which is 2.4 GHz. The rest of the requirements belongs to the physical part of the quadcopter. Sixth, weight needs to be under 500g (17.6oz). Seventh, the quadcopter should and only can be powered by the electric battery. Eighth, the quadcopter need to be smaller than 45 cm (17.7 inches) in any direction. Safety concern of using the quadcopter builds the last requirement. The remote control of the quadcopter has a Kill Switch buttom which can kill the power of moving quadcopter and put it in the static state immediately.  
Before starting to build a quadcopter, each team needs to provide information for applying competition. Overhead design including team number and specialist for each member, onboard system and remote operation around January (according to 2016 and previous competitions' rules).  
After the first selection of designs and Paper submission, the entrance admission becomes strict. The next step for improvement is to send a video of the in-process quadcopter. It is not a simple video, line-of-sight operation, real-time take-off, real-time grounding and relevant feedback of the sensor will provide as necessary information. Information received from sensors and remoter will be analyzed by our team.  
\section{The Solution Section}
Our team will use all the data as a bridge to help the quadcopter fly stable. The next step for quadcopter is an opportunity to show audiences how it can work. The quadcopter has the excellent performance of finishing the different and challenging mission. It is more accessible to remote control by a human. However, these similar assignments will be completed by only camera detection. Incoming data, it is easy to be blocked by wood, or any size of structures made of thick material.  
An efficient way to prevent that situation happened is to make sure transmitter and the signal receiver in the right connected area. Our team will develop software or applications to calculate the tunnel of the destination. During the flying time for reaching the goal, the shaft will have a target will we need to search, and several obstacles to avoid. One of the main of the software is being able to recognize the mission target and find it in the way of destination. For the safety of the flying task, the quadcopter also needs the software can distinguish the difference between objective and the obstacles. According to the requirements of the mission, the software is going to provide a function of recognizing essential objects.  
\section{The Research Section}
It is a challenge to make a program which has a function for recognizing objects. Human learns an object by knowing theirs' functions and features. However, it is hard for a program to understand what is the affection of the object only from the picture. Our team is planning to use sensors and cameras to detect the distance and the speed we need to apply on the quadcopter. In the market, there are some existing methods to control the quadcopter. Arduino is a well-known and easy way to control the quadcopters. However, Arduino can only use in their product; it will limit the choice we can choose. There are many other libraries and open source lab on the Internet. This kind of information will provide and enhance the code we are going to program for the quadcopter. Like git, python or many other different open source labs are the beneficial tools for real-time camera analysis and solving environment condition. Objects' distinctions and Obstacles avoiding will be the biggest issue for the program. The reflection time cannot be too long; otherwise, the quadcopter will timeout or run into low battery condition and then crash land. Colors distinct can help computer recognize and eliminate useless objects. The changing sizes of images can help the computer analysis the speed of moving and distance of tunnel. 2.4GHz is the standard communication between quadcopter and the computer; our team is going to learn and use video data transmitter to build the quadcopter. 
\section{The Final Section}
However, in real life, the quadcopter is limited by a lot of changing environment features and technologies. When driving a quadcopter, it can be affected by high wind speed, which can cause control lost or crash damage on the ground. During that situation, the video information will not be detail and precise. It is going to have problems for engineers to find contest from the video. In another case like rainy weather, the electric power quadcopter is going to have significant damage. In this position, the video recorder from quadcopter may not be able to work correctly.  
After our team has reached the requirements for entering the competition, all of the situations above will be considered and provide solutions. 

\end{document}
