\documentclass[10pt,letterpaper,onecolumn]{article}

\usepackage{color}
\usepackage{url}
\usepackage{hyperref}
\usepackage{enumitem}
\usepackage[margin=0.75in]{geometry}



\begin{document}
\begin{titlepage}
\begin{center}
    \huge
    \textbf{Problem Statement}

    \vspace{0.4in}
    \large
    Instructor: D. Kevin McGrath, Kirsten Winters\\
    Sponsor: Nancy Squires, Ph.D

    \vspace{0.2in}
    \large
    CS461 - CS Senior Capstone\\
    AHS Micro-Air Vehicle Challenge\\
    Fall 2017

    \vspace{0.2in}
    \textbf{Kaiyuan Fan , Yingshi Huang , Justin Sherburne }

    \vspace{0.5in}
    \textbf{Abstract}\\
    \vspace{0.2in}
    \end{center}
American Helicopter Society invites student teams to participate in the 5th Annual Micro Air Vehicle (MAV) Student Challenge. Our project is to design a quadcopter that is capable of competing in this event on May 14-17. This project is sponsored by Nancy Squires Ph.D., in collaboration with Columbia Helicopters. Throughout this project, we will be working in close coordination with a mechanical engineering team and an electrical engineering team to ensure that our micro-air vehicle can meet the competition requirements. In order to compete our vehicle must weigh less than 500 grams, be shorter than 17.7 inches in any dimension, maintain a stable flight path, and have an on-board camera system. The goal of the project is to create an autonomous micro-air vehicle capable of meeting the competition requirements.

    \vspace{0.3in}
\end{titlepage}


\newpage

\begin{center}
\large
\textbf{Problem Description}
\end{center}

The Micro-Air Vehicle challenge is an annual collegiate competition hosted by The American Helicopter Society. The competition is conducted by the AHS International’s Unmanned VTOL Aircraft and Rotorcraft Committee. Competitors are tasked with building a vehicle capable of steady-state hovering, avoiding obstacles, and recognizing a dropoff point. This vehicle may compete in either a manual or autonomous flight category. The aircraft is remotely controlled and will have the ability to deliver packages to a specified location. While the 2018 competition rules have not yet been announced, last year’s competition included the restrictions on size, weight, and movement listed below:
\begin{itemize}
\item Vertical takeoff and landing (VTOL) as well as hover capability
\item Any number of rotors/propellers
\item Onboard flight-stabilization
\item Onboard camera(s) needed for mission – Multiple cameras are allowed.
\item Standard communication (preferred 2.4 GHz)
\item Weight less than 500g (17.6 oz) including batteries (not including the delivery packages)
\item Size less than 45 cm (17.7 inches) in any dimension
\end{itemize}

We are creating a stable and reliable air vehicle. The aircraft must be fully tested before competition to ensure each function works reliably. An emergency shutdown and manual override function is required in case of unacceptable aircraft behavior. Besides the implementation of Micro Air Vehicle, we also need to present our design at the competition to a professional audience. The presentation is supposed to illustrate the design process of the air vehicle, explain autonomous elements and package delivery system. The presentation should include a short flight demo to illustrate the capabilities of the aircraft for the audience.


\begin{center}
\large
\textbf{Proposed Solution}
\end{center}

We are going to design and manufacture a quadcopter to compete in the micro air vehicle challenge. This is because quadcopters are cheaper and more durable than conventional helicopters. Additionally their smaller blades are more advantageous because they possess less kinetic energy. For small-scale quadcopters, this makes the vehicles safer for close interaction. Quadcopters differ from conventional helicopters, which use rotors that can vary the pitch of their blades dynamically as they move around the rotor hub. As the computer science team in this project, we are tasked with programming the quadcopter system with sufficient functionalities to enable semi-autonomous or fully-autonomous flight. In addition to flight capabilities, we will also be transmitting flight information back to a remote station. This data will be converted into usable information in the form of graphs, spreadsheets, and potentially three-dimensional visualization of the flight path. In addition to the programming requirements, we have a responsibility to communicate and collaborate with the electrical engineering and mechanical engineering team. We will need to convey our requirements for sensors and processing hardware needed to meet our goals.\\

\begin{center}
\large
\textbf{Performance Metrics}
\end{center}

Our team will draft a design proposal before the end of the January. This design proposal will be submitted to the AHS International Unmanned VTOL Aircraft and Rotorcraft Committee. The framework electronic modules of aircraft will be designed and manufactured this fall term. Winter term we will focus on building and refining our vehicle for autonomous flight. Testing and debugging will begin at least a month prior to the competition. Before mid-March, we must submit a test video of your aircraft to the committee for consideration. If our team is selected we will attend the final competition in May at Phoenix. After the competition, we will attend the undergraduate engineer expo to share the experiences and present the design and manufacturing processes.\\
Success will be measured on the following performance metrics:

\begin{enumerate}
\item Vehicle Completion:
\begin{itemize}
\item Can the vehicle fly under its own power?
\item Does it weigh less than 500g?
\item Is the vehicle shorter than 17.7 inches in any dimension?
\item Is the vehicle able to maintain a stable flight path?
\end{itemize}
\item Autonomous Completion:
\begin{itemize}
\item Can the vehicle stream video to a remote station?
\item Can the vehicle be controlled remotely from a computer?
\item Can the vehicle follow a specific flight plan without human interaction?
\item Can the vehicle locate targets and choose a flight plan without human
interaction?
\end{itemize}
\item Competiton Completion:
\begin{itemize}
\item The team must submit an intent to compete in the 2018 AHS Mirco-Air Vehicle
Challenge.
\item The team must submit a design proposal for competition before the mid-
January cut-off date.
\item The team must demonstrate competition readiness by meeting the vehicle and
autonomous requirements prior to the competition date, and submit video evidence
to the competition organizers for final selection.
\item If selected to compete in the 2018 challenge, the team must travel to
Phoenix to participate in the challenge.
\end{itemize}
\end{enumerate}

\section{Reference}
MAV Student Challenge 2017. (n.d.). Retrieved October 10, 2017, from https://vtol.org/education/micro-air-vehicle-student-challenge/micro-air-vehicle-student-challenge-2017\\
Quadcopter. (2017, October 07). Retrieved October 10, 2017, from https://en.wikipedia.org/wiki/Quadcopter

\end{document}
